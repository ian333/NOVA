\documentclass[a4paper,12pt]{article}
\usepackage{fontspec}
% \usepackage[spanish]{babel} % Spanish language support not available in environment

% Define custom font for TOC
\newfontfamily\shrikhand[Path=assets/fonts/]{Shrikhand-Regular.ttf}

\renewcommand{\contentsname}{\textcolor{blue}{\shrikhand Contenido}} % Custom TOC title
\usepackage{graphicx}
\usepackage{hyperref}
\usepackage{listings}
\usepackage{xcolor}
\usepackage{geometry}

% Configuration for code listings
\definecolor{codegreen}{rgb}{0,0.6,0}
\definecolor{codegray}{rgb}{0.5,0.5,0.5}
\definecolor{codepurple}{rgb}{0.58,0,0.82}
\definecolor{backcolour}{rgb}{0.95,0.95,0.92}

\lstdefinestyle{mystyle}{
    backgroundcolor=\color{backcolour},   
    commentstyle=\color{codegreen},
    keywordstyle=\color{magenta},
    numberstyle=\tiny\color{codegray},
    stringstyle=\color{codepurple},
    basicstyle=\ttfamily\footnotesize,
    breakatwhitespace=false,         
    breaklines=true,                 
    captionpos=b,                    
    keepspaces=true,                 
    numbers=left,                    
    numbersep=5pt,                  
    showspaces=false,                
    showstringspaces=false,
    showtabs=false,                  
    tabsize=4
}

\lstset{style=mystyle}
\geometry{margin=2.5cm}

\begin{document}

\begin{titlepage}
    \centering
    \vspace*{\fill}
    
    % Imagen de portada (Portada completa)
    \includegraphics[width=\textwidth,height=0.95\textheight,keepaspectratio]{assets/portada}
    
    \vspace*{\fill}
\end{titlepage}

%\tableofcontents
%\newpage

\section*{\textcolor{blue}{\shrikhand Bienvenido}}
\addcontentsline{toc}{section}{Bienvenido}

Tal vez pienses en las computadoras como esos dispositivos con pantalla y teclado que tienes en tu escritorio, y es cierto, lo son. Pero no son el único tipo.

En esta guía exploraremos los microcontroladores: pequeñas unidades de procesamiento que seguramente ya tienes en casa sin saberlo. Es muy probable que tu lavadora, microondas o termostato estén controlados por uno. Sin embargo, esos dispositivos suelen ser cerrados: sus fabricantes dificultan o impiden que modifiques el software que ejecutan.

\textbf{NOVA}, por otro lado, es diferente. Se puede reprogramar fácilmente todas las veces que quieras a través de una simple conexión USB. En las siguientes páginas veremos cómo empezar a utilizar este hardware y cómo combinarlo con otros componentes electrónicos.

Lo que construyas con ellos depende totalmente de ti.

Esta es la guia de \textbf{NOVA}. Esta documentación está diseñada para hobbistas, estudiantes y entusiastas que quieren entender \textbf{cómo funcionan las cosas} sin perderse en tecnicismos matemáticos complejos.

Si tu objetivo es aprender electrónica programable, robótica básica y control de dispositivos, este es tu punto de partida.

\newpage

\section*{\textcolor{blue}{\shrikhand Contenidos}}

\textcolor{red}{\textbf{Capítulo 1: Conoce tu NOVA}}\\
Familiarízate con tu nuevo y potente microcontrolador, aprende cómo usar sus pines e instala MicroPython para programarlo.

\textcolor{red}{\textbf{Capítulo 2: Programación con MicroPython}}\\
Conecta una computadora y comienza a escribir programas para tu NOVA usando el lenguaje MicroPython.

\textcolor{red}{\textbf{Capítulo 3: Computación Física}}\\
Aprende sobre los pines de tu NOVA y los componentes electrónicos que puedes conectar y controlar.

\textcolor{red}{\textbf{Capítulo 4: Computación Física con NOVA}}\\
Empieza a conectar componentes electrónicos básicos a tu NOVA y escribe programas para controlarlos y detectar sus señales.

\textcolor{red}{\textbf{Capítulo 5: Controlador de Semáforos}}\\
Crea tu propio mini sistema de cruce peatonal usando múltiples LEDs y un botón pulsador.

\textcolor{red}{\textbf{Capítulo 6: Juego de Reacción}}\\
Construye un juego simple de tiempos de reacción usando un LED y botones, para uno o dos jugadores.

\textcolor{red}{\textbf{Capítulo 7: Alarma Antirrobo}}\\
Usa un sensor de movimiento para detectar intrusos y activa la alarma con una luz intermitente y una sirena.

\textcolor{red}{\textbf{Capítulo 8: Medidor de Temperatura}}\\
Usa el ADC integrado de NOVA para convertir entradas analógicas y lee su sensor de temperatura interno.

\textcolor{red}{\textbf{Capítulo 9: Registrador de Datos (Data Logger)}}\\
Convierte tu NOVA en un dispositivo de registro de temperatura y desconéctalo de la computadora para hacerlo totalmente portátil.

\textcolor{red}{\textbf{Capítulo 10: Protocolos de Comunicación Digital: I2C y SPI}}\\
Explora estos dos protocolos de comunicación populares y úsalos para mostrar datos en una pantalla LCD.

\newpage

\section{Conoce tu NOVA}

NOVA es una herramienta potente. En su corazón late el \textbf{Raspberry Pi RP2350A}, un microcontrolador moderno.

\subsection{¿Qué significa esto en la práctica?}
\begin{itemize}
    \item \textbf{Doble Núcleo (Dual Core):} Imagina que tienes dos cerebros. Mientras uno se encarga de mantener encendida una pantalla, el otro puede estar leyendo sensores o controlando un motor. Trabajan en paralelo (Multitasking real).
    \item \textbf{Velocidad (150 MHz):} Puede procesar millones de instrucciones por segundo. Para proyectos de luces, robots y sensores, es increíblemente rápido.
    \item \textbf{Pines de Propósito General (GPIO):} Son las "patitas" de la placa. A diferencia de un puerto USB normal de tu PC, estos pines se pueden programar para ser interruptores (encender/apagar cosas) o entradas (leer botones/sensores).
\end{itemize}

\subsection{Componentes principales de tu NOVA}

Para dominar NOVA, primero debemos llamar a cada cosa por su nombre. Aquí tienes el "diccionario" de tu hardware:

\begin{figure}[ht!]
    \centering
    \includegraphics[width=0.85\textwidth]{assets/Componentes.png}
    \caption{Mapa de componentes de NOVA}
\end{figure}

\begin{itemize}
    \item \textbf{Puerto USB-C (Energía y Datos):} Es el cordón umbilical de la placa. Por aquí recibe electricidad (5V) para funcionar y datos desde tu computadora cuando la programas.
    
    \item \textbf{Chip RP2350 (El Microcontrolador):} Es el "cerebro" real. Es el componente cuadrado y negro en el centro. Aquí es donde "vive" tu código y se toman las decisiones a 150 Mhz.
    
    \item \textbf{Memoria Flash (Almacenamiento):} Es como el disco duro de tu computadora, pero en miniatura. Aquí se guarda tu programa para que no se borre cuando desconectas la placa.
    
    \item \textbf{GPIOs (Pines de Entrada/Salida):} Esas "patitas" o agujeros dorados a los lados. Son las manos y ojos de NOVA. Puedes configurarlos para \textbf{leer} el mundo (sensores) o \textbf{actuar} sobre él (encender luces, mover motores).
    
    \item \textbf{Botón BOOT (Modo de carga):} Si lo mantienes presionado al conectar el USB, le dices a NOVA: "¡Alto! No ejecutes nada, voy a instalarte un nuevo sistema".
    
    \item \textbf{Botón RESET (Reinicio):} Es como desenchufar y volver a enchufar la placa, pero más rápido. Útil si tu programa se queda atascado.
    
    \item \textbf{LED de Usuario (Test):} Una pequeña luz integrada en la placa que tú puedes controlar. Es perfecta para realizar tus primeras pruebas sin conectar nada extra.
\end{itemize}

\clearpage
\section{Programación con MicroPython}

Para comunicarnos con NOVA, usaremos \textbf{MicroPython}.
\begin{itemize}
    \item \textbf{¿Qué es?} Es una versión de Python 3 optimizada para funcionar en microcontroladores. Es el lenguaje ideal para empezar porque es legible y potente.
    \item \textbf{La Herramienta:} Usaremos \textbf{Thonny IDE}. Es un editor de código simple que ya trae todo lo necesario para hablar con tu placa.
\end{itemize}

\subsection{Pasos rápidos:}
\begin{enumerate}
    \item Conecta tu NOVA al ordenador manteniendo pulsado el botón \textbf{BOOT}.
    \item Tu PC lo verá como una memoria USB.
    \item Descarga la \textbf{última versión} del firmware \texttt{.uf2} para RP2350 desde \url{https://micropython.org/download/rp2-pico2/}. Copia o arrastra el archivo descargado en esa memoria. (La placa se reiniciará automáticamente, eso es normal).
    \item Abre Thonny y configura el intérprete.
    \begin{itemize}
        \item \textbf{Ubicación:} Haz clic en el texto que aparece en la \textbf{esquina inferior derecha} de la ventana principal (o ve al menú \textit{Ejecutar} $\rightarrow$ \textit{Configurar intérprete}).
        \item \textbf{Selección:} Elige la opción \textbf{"MicroPython (Raspberry Pi Pico)"}.
        \item \textbf{¿Por qué este?} Aunque tu placa se llama NOVA, su cerebro (RP2350) pertenece a la familia RP2. Thonny necesita este controlador específico para poder comunicarse por USB, enviar tu código y mostrar los mensajes de la placa en la consola.
    \end{itemize}

    \vspace{0.2cm}
    \makebox[\linewidth][c]{
        \includegraphics[width=\textwidth]{assets/previews/thonny_hello_1.png}
    }
    \begin{center}
        \vspace{-0.3cm}
        \small{\textit{Interfaz de Thonny con el primer código}}
    \end{center}
\end{enumerate}

\subsection{Tu primer programa: Hola Mundo (Blink)}
En el mundo de la programación de software, lo primero que aprendemos es a imprimir un texto en pantalla. En el mundo de los microcontroladores, nuestro "Hola Mundo" es hacer parpadear un LED.

Copia el siguiente código en el editor principal de Thonny:

\begin{lstlisting}[language=Python]
from machine import Pin
import time

# Configuramos el LED integrado (Pin 25)
led = Pin(25, Pin.OUT)

while True:
    led.toggle()      # Cambia estado (ON/OFF)
    print("Blink!")   # Mensaje en consola
    time.sleep(1)     # Espera 1 segundo
\end{lstlisting}

\subsubsection{Explicación del código:}
\begin{enumerate}
    \item \textbf{Importar librerías:} \texttt{machine} nos da control sobre el hardware y \texttt{time} maneja el tiempo.
    \item \textbf{Configuración:} Creamos un objeto \texttt{led} en el pin 25 y le indicamos que será de \textbf{SALIDA} (OUT), es decir, enviará voltaje.
    \item \textbf{Bucle Infinito:} \texttt{while True:} crea un ciclo sin fin para que el programa nunca se detenga.
    \item \textbf{Acción:} \texttt{.toggle()} invierte el estado actual del LED.
\end{enumerate}

Pulsa el botón \textbf{Ejecutar} (ícono de "Play" verde) en Thonny. ¡Verás el LED de tu NOVA parpadear!

\vspace{1cm}

\begin{center}
    \textbf{\Large \textcolor{blue}{¡Felicidades!}}\\
    \vspace{0.2cm}
    Has logrado programar tu primer código y encender el LED correctamente. ¡Este es tu primer gran paso en el control de hardware!
\end{center}

\vspace{0.5cm}

\textbf{Por cierto, ¿sabes lo que es hardware, firmware y software? ¿No? Bueno, te explico rápido:}

Imagina que tu NOVA es un músico.
\begin{itemize}
    \item El \textbf{Hardware} es el instrumento (la guitarra, el piano). Es la parte física, lo que puedes tocar y, si se cae, se rompe.
    \item El \textbf{Firmware} es el \textbf{traductor experto}. Tu placa solo entiende de electricidad (voltajes), y tú escribes palabras en inglés (Python). El firmware es el programa que vive en el chip y se encarga de traducir tus órdenes humanas a acciones eléctricas reales.
    \item El \textbf{Software} es el \textbf{director de orquesta} (¡Ese eres tú!). Es el código lógico que escribes en Thonny para decidir cuándo y cómo debe sonar el instrumento. Sin tu software, el hardware está mudo y el firmware aburrido.
\end{itemize}
¡Tú acabas de componer tu primera obra!

\newpage
\section{Hardware y Conexiones}

Aquí entramos en el mundo de la \textbf{Electrónica}: conectar el software con el mundo real. Hasta ahora, tu código vivía encerrado en el chip; ahora vamos a darle sentidos (sensores) y músculos (motores/luces).

\subsection{El Mapa de tu NOVA (Pinout)}
Para conectar componentes sin "quemar" nada, necesitas un mapa. A los pines metálicos los llamamos \textbf{GPIO} (\textit{General Purpose Input/Output}). NO todos son iguales, así que presta atención a este diagrama:

\begin{figure}[h!]
    \centering
    \includegraphics[width=\textwidth]{assets/pin out.png}
    \caption{Diagrama de pines (Pinout) de NOVA RP2350}
\end{figure}

\subsection{Tipos de Pines y su función}
\begin{itemize}
    \item \textbf{GP0 - GP29 (Digitales):} Son los pines "normales". Solo entienden dos estados: \textbf{3.3V} (Encendido/1) y \textbf{0V} (Apagado/0).
    \item \textbf{GND (Tierra/Masa):} ¡El más importante! Es el polo negativo. Todos los circuitos electrónicos deben conectarse a GND para cerrarse y funcionar. En NOVA, los verás marcados en negro o con formas cuadradas en el diagrama.
    \item \textbf{3V3 (Energía):} Estos pines siempre entregan 3.3 voltios. Úsalos para "alimentar" sensores o componentes que necesiten energía constante.
    \item \textbf{ADC (Analógicos):} Pines especiales (GP26, GP27, GP28) que pueden medir voltajes variables (como el volumen de una perilla o la intensidad de luz), no solo "todo o nada".
\end{itemize}

\subsection{Protocolos de Comunicación (Para el futuro)}
Además de encender luces, algunos pines saben "hablar" idiomas complejos para conectarse con sensores avanzados o pantallas:
\begin{itemize}
    \item \textbf{I2C:} Usa 2 cables (SDA, SCL). Ideal para pantallas OLED y sensores de movimiento.
    \item \textbf{UART:} Comunicación serie clásica. Útil para módulos GPS o Bluetooth.
    \item \textbf{SPI:} Comunicación ultra-rápida. Usada en lectores de tarjetas SD o pantallas a color.
\end{itemize}

\newpage
\section{Control Digital: Luces y Botones}

En el capítulo anterior vimos el mapa. Ahora vamos a manejar el coche.
Los pines GPIO digitales son la forma más básica de comunicación: solo entienden dos palabras: \textbf{ENCENDIDO (1/High)} y \textbf{APAGADO (0/Low)}.

\subsection{Salidas Digitales (Escribir)}

Cuando configuras un pin como \textbf{SALIDA (OUT)}, el microcontrolador "impone" un voltaje. Es como abrir o cerrar un grifo de agua.

Vamos a controlar un LED externo. A nivel eléctrico, los LEDs son delicados; siempre necesitan una \textbf{resistencia de protección} (220$\Omega$ o 330$\Omega$) para no quemarse por exceso de corriente.

\textbf{Circuito:}
\begin{itemize}
    \item \textbf{Pin 15} $\rightarrow$ Pata Larga del LED (+).
    \item \textbf{GND} $\rightarrow$ Resistencia $\rightarrow$ Pata Corta del LED (-).
\end{itemize}

\textbf{Código Básico (Parpadeo):}
\begin{lstlisting}[language=Python]
from machine import Pin
import time

led = Pin(15, Pin.OUT) # Configuramos Pin 15 como SALIDA

while True:
    led.toggle()      # Cambia el estado (si estaba on, se apaga)
    time.sleep(0.5)
\end{lstlisting}

\subsection{Proyecto: Ojos de Robot (Lógica)}

Vamos a subir el nivel. Usaremos el módulo \texttt{random} y funciones de Python para dar "personalidad" a dos LEDs, simulando ojos que parpadean o escanean.

\begin{lstlisting}[language=Python]
from machine import Pin
import time
import random

# Definimos los pines de los LEDs
ojo_izq = Pin(15, Pin.OUT)
ojo_der = Pin(14, Pin.OUT)

def parpadeo_normal():
    """Simula un parpadeo rapido simultaneo"""
    ojo_izq.off()
    ojo_der.off()
    time.sleep(0.15)
    ojo_izq.on()
    ojo_der.on()

def escaneo():
    """Alterna luces como buscando algo"""
    print("Escaneando entorno...")
    for i in range(4):
        ojo_izq.toggle()
        time.sleep(0.08)
        ojo_der.toggle()
        time.sleep(0.08)
    # Al finalizar, encendemos ambos
    ojo_izq.on()
    ojo_der.on()

# --- Bucle Principal ---
print("Sistema Iniciado. Ejecutando IA basica.")
ojo_izq.on()
ojo_der.on()

while True:
    # Espera aleatoria entre acciones (2 a 5 segundos)
    time.sleep(random.uniform(2.0, 5.0))
    
    # 70% Probabilidad de parpadeo, 30% escaneo
    if random.randint(1, 100) <= 70:
        parpadeo_normal()
    else:
        escaneo()
\end{lstlisting}

\subsection{Entradas Digitales (Leer Botones)}
Hasta ahora NOVA solo "hablaba". Ahora va a "escuchar".
Para leer un botón, configuramos el pin como \textbf{ENTRADA (IN)}.

Pero hay un problema: ¿Qué voltaje hay en el cable cuando el botón NO está presionado? ¿0V? ¿3.3V? La respuesta es radiación electromagnética aleatoria (ruido).
Para evitar falsas lecturas, activamos una resistencia interna llamada \textbf{PULL\_DOWN}, que conecta el pin a tierra (0V) suavemente cuando nadie lo toca.

\textbf{Reto:} Crea un interruptor de luz.
\begin{lstlisting}[language=Python]
from machine import Pin
import time

led = Pin(15, Pin.OUT)
boton = Pin(16, Pin.IN, Pin.PULL_DOWN) # PULL_DOWN asegura 0V si no pulsas

print("Listo para recibir ordenes...")

while True:
    estado = boton.value() # Leemos el estado (0 o 1)
    
    if estado == 1:
        led.on()
    else:
        led.off()
        
    time.sleep(0.05) # Pequeña pausa para estabilidad
\end{lstlisting}

\subsection{Movimiento y PWM (Servomotores)}

Un servomotor no funciona simplemente "encendiendo y apagando" la corriente. Necesita una \textbf{señal de control precisa} usando PWM (Pulse Width Modulation).

\textbf{Código de Control de Servo:}

\begin{lstlisting}[language=Python]
from machine import Pin, PWM
import time

# Configuracion del PWM en el Pin 16
# Los servos estandar funcionan a una frecuencia de 50Hz
servo = PWM(Pin(16))
servo.freq(50)

def mover_servo(angulo):
    """
    Convierte un angulo (0-180) al ciclo de trabajo (duty) necesario.
    """
    # Mapeo simple: 0-180 -> duty_u16
    duty = int(1638 + (angulo / 180) * (8192 - 1638))
    servo.duty_u16(duty)

while True:
    print("Moviendo a 0 grados")
    mover_servo(0)
    time.sleep(1)
    
    print("Moviendo a 90 grados")
    mover_servo(90)
    time.sleep(1)
    
    print("Moviendo a 180 grados")
    mover_servo(180)
    time.sleep(1)
\end{lstlisting}

\hrulefill

\textit{Nota: Los capítulos siguientes (5 al 10) están en desarrollo para futuras actualizaciones.}

\section*{Recursos Adicionales}
\addcontentsline{toc}{section}{Recursos Adicionales}

\begin{itemize}
    \item \textbf{Documentación Oficial MicroPython:} \href{https://docs.micropython.org}{docs.micropython.org}
    \item \textbf{Datasheet RP2350:} Para cuando necesites los detalles técnicos profundos de los registros y memoria.
    \item \textbf{Repositorio del Proyecto:} Ejemplos de código y librerías específicas para NOVA.
\end{itemize}

¡Bienvenido al nivel intermedio! Aquí es donde la creatividad se encuentra con la ingeniería.

\end{document}
